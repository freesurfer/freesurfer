\documentclass[10pt]{article}
\usepackage{amsmath}
%\usepackage{draftcopy}

%%%%%%%%%% set margins %%%%%%%%%%%%%
\addtolength{\textwidth}{1in}
\addtolength{\oddsidemargin}{-0.5in}
\addtolength{\textheight}{.75in}
\addtolength{\topmargin}{-.50in}

%%%%%%%%%%%%%%%%%%%%%%%%%%%%%%%%%%%%%%%%%%%%%%%%%%%%%%%%%%%%%%%%%
%%%%%%%%%%%%%%%%%%%%%%% begin document %%%%%%%%%%%%%%%%%%%%%%%%%%
%%%%%%%%%%%%%%%%%%%%%%%%%%%%%%%%%%%%%%%%%%%%%%%%%%%%%%%%%%%%%%%%%
\begin{document}

\begin{Large}
\noindent {\bf stxgrinder} \\
\end{Large}

\noindent 
\begin{verbatim}
Comments or questions: analysis-bugs@nmr.mgh.harvard.edu
\end{verbatim}

\section{Introduction}
{\bf stxgrinder} is a program for generating significance maps for
fMRI experiments processed with the {\bf selxavg} selective averaging
program.  Requires matlab 5.2 or higher.\\

Note: in previous versions (prior to 3/1/2000), stxgrinder would, by
default, report the natural log of the significance values.  It now
defaults to log10 of the significance values.

\section{Usage}
Typing {\em stxgrinder} at the command-line without any options will give the
following message:\\ 

\begin{small}
\begin{verbatim}
USAGE: stxgrinder [-options] -i instem -o outstem
   instem   - prefix of .bfloat selxavg files
   outstem  - prefix of .bfloat output files
Options:
   -cmtx cmtxfile        : contrast matrix file (see mkcontrast)
   -a a1 [-a a2 ...]     : active condition(s)
   -c c1 [-c c2 ...]     : control condition(s)
   -delrange dmin dmax   : delay range to test <prestim,timewindow> (sec)
   -ircorr               : correlate with ideal HDIR
   -delta                : delta of ideal HDIR
   -tau                  : tau of ideal HDIR
   -testtype             : stat test type (t,<tm>,Fd,Fc,Fcd)
   -pmin                 : get min p value from tm test over delrange
   -format string        : lnp, <log10p>, p, tF
   -monly mfile          : do not run, just create matlab file
   -firstslice <int>     : first slice to process <auto>
   -nslices <int>        : number of slices to process <auto>
   -version              : print version and exit
\end{verbatim}
\end{small}

\section{Command-line Arguments}

\noindent
{\bf -i instem}: stem of the volume of hemodynamic averages (in {\em
bfile format}). This is the output of {\bf selxavg} .\\

\noindent
{\bf -o outstem}: this is the stem of the output significance volume.\\

\noindent
{\bf -cmtx cmtxfile}: specify a contrast matrix file as created by
{\bf mkcontrast}.  A cmtxfile is specified in place of the following 
options: -a, -c, -delrange, -ircorr, -delta, and -tau.\\

\noindent
{\bf -a a1 [-a -a2 ...]}: the active condition(s) (or positive
contrast condition) for the statistical test where {\em a1} is an
integer that uniquely identifies the the first condition (as specified
in the paradigm file).  There must be at least one active condition
specified.  All the active conditions are summed together, and the sum
of the control conditions is subtracted from the sum of the active
conditions.  Eg, {\em -a 1 -a 4}.\\

\noindent
{\bf -c c1 [-c -c2 ...]}: the control condition(s) (or negative
contrast condition) for the statistical test where {\em c1} is an
integer that uniquely identifies the the first condition.  There must
be at least one control condition specified.  All the control
conditions are summed together and subtracted from the sum of the
active conditions. Eg, {\em -c 0 -c 3}.\\

\noindent
{\bf -delrange dmin dmax}: set the range of hemodyanmic delays over
which to perform the statistical test.  {\em dmin} is the minimum
delay, and {\em dmax} is the maximum delay; both are specified in
seconds.  If {\em -delrange} is not specified, then all time points in
the hemodynamic response are considered. This option cannot be used
when the {\em -gammafit} option was used during selective averaging.\\

\noindent
{\bf -ircorr}: instruct {\em stxgrinder} to correlate the hemodynamic
averages with an ``ideal'' hemodynamic impulse response.  The ideal
is a gamma function which has two parameters: $\Delta$ and $\tau$,
which are specified with their own flags.
The gamma function has the form:
\begin{equation}
h(t) = 
\begin{cases}
0 & t < \Delta \\
(\frac{\tau e^2}{4})
( \frac{t-\Delta}{\tau} )^2 e^{{-(\frac{t-\Delta}{\tau}})} & t > \Delta \\
\end{cases}
\label{hideal.eqn}
\end{equation}
$\Delta$ is the time (in seconds) between stimulus onset and the start
of the time at which blood flow starts to increase.  $\tau$ controls
how fast the HDR rises and falls.  The factor $\frac{\tau e^2}{4}$
forces the peak of the response to be 1.0.  This option cannot be
used when the {\em -gammafit} option was used during selective averaging.\\

\noindent
{\bf -delta double}: This specifies the delay of the ``ideal''
hemodynamic impulse response in seconds (see {\bf -ircorr}).
A typical value is 2.25 sec. Eg: {\em -delta 2.25}.\\

\noindent
{\bf -tau double}: This specifies the time constant of the ``ideal''
hemodynamic impulse response in seconds (see {\bf -ircorr}).
A typical value is 1.25 sec. Eg: {\em -tau 1.25}.\\

\noindent
{\bf -testtype string}: this flag indicates the type of statistical
test to perform.  Valid strings are: {\em t, tm, Fd, Fc, Fcd}. Tests
that begin with a {\em t} us a (univariate) t test. Tests that begin
with an {\em F} us a (multivariate) F test. Modifiers {\em m, d, c,}
and {\em cd} determine how time-points and conditions are collapsed
together (see below).\\

\noindent
{\bf -pmin }: this flag, used with testtype {\em tm} which perform a
different test on each time point, so there are multiple significance
values for each voxel.  This flag instructs {\em stxgrinder} to search
each voxel over time points for its minimum p-value.  This value is
then multiplied by the number of time points searched (ie, a Bonferoni
correction), and this one significance value is reported. This is a
way to do a multivariate test from a series of univariate tests.\\

\noindent
{\bf -format string}: specifies the format of the output.  Allowable
choices are {\em lnp, log10, p, tF}.  {\em lnp} specifies that the
natural log significance values be stored.  {\em log10} instructs
{\em stxgrinder} to store the base-10 log of the significance to
be stored.  Using {\em p} forces the raw p-values to be stored.
{\em tF} specifies that the statistic value (either from the t or
F test) be stored instead of computing a significance value from
the statistic. Eg: {\em -format log10p}.\\

\noindent
{\bf -monly}: only generate the matlab file which would accomplish the
analysis but do not actually execute it.  This is mainly good for
debugging purposes.\\

\noindent
{\bf -firstslice int}: first {\em anatomical} slice to process
(usually 0). Autodetcted is left unspecified.\\

\noindent
{\bf -nslices int}: total number of anatomical slices to process.
Autodetcted is left unspecified.\\

\section{Output}

{\em stxgrinder} will create a statistical map for each anatomical
slice in the volume specified.  The file name will be
``outstem\_sss.bfloat'' where ``sss'' is the anatomical slice number.
This map will either have one plane or will have the same number of
planes as the number of delays specified in the {\em -delrange}
option, depending upon which {\em testtype} is specified.  The {\em
tm} and {\em Fm} types can generate multiple planes; the rest can only
generate a single plane.  If the {\em pmin} flag is set with the {\em
tm} or {\em Fm} options, only one plane is reported.  The significance
value can either be the raw p value, the natural log of the p value,
or the base 10 log of the p value, depending upon the choice of the
{\em format} flag.  The t or F statistic itself can also be reported
instead of the p value.  For t test, the significance will have the
same sign as the underlying t value.

\section{Specifying Hypotheses}

{\em stxgrinder} allows the user to test a multitude of hypotheses.
This section describes how to determine the command-line arguments
that will realize a given hypothesis.  This will be done in two
parts: specifying a restriction table and specifying a test type.

\subsection{Specifying the Restriction Table}

Regardless of the hypothesis, a {\em single} number (the statistic) is
computed and tested for significance.  The statistic is computed from
the hemodynamic averages and variances.  Each point within the
hemodynamic time window (specified when {\em selxavg} is run) has its
own average and variance.  Each condition has its own set hemodynamic
averages and variances.  If there are $N_c$ conditions with $N_h$
hemodynamic points per condition, then there will be $N_{ch} = N_h *
N_c$ averages and variances that can be combined in any way to create
a statistical test.  The number of combinations and permutations is
potentially enormous, and {\em stxgrinder} restricts the user to a
subset of tests that are most useful for the task at hand.  To see how
this works, it is helpful to represent all the hemodynamic points
within a table or matrix with the columns representing the range of
post-stimulus delays and the rows representing the different
conditions as shown below for an experiment with three conditions and
a time window of 7 TRs.\\

\begin{tabular}{|r||r|r|r|r|r|r|r|r|}\hline
& 0 TR & 1  TR & 2  TR & 3  TR & 4  TR & 5  TR & 6  TR & 7  TR\\ \hline \hline
Condition 0 &   &    &    &    &    &    &    &   \\ \hline
Condition 1 &   &    &    &    &    &    &    &   \\ \hline
Condition 2 &   &    &    &    &    &    &    &   \\ \hline
\end{tabular}\\

To specify the components to test, one fills the slots of the table
with zeros and $\pm 1$'s.  One first decides on the conditions to be
tested and over which delays they will be tested.  Conditions not to
be tested should have zeros in their rows.  Delays not to be tested
should have zeros in their columns.  A remaining slot should be
filled with $+1$'s if its condition is active or $-1$'s if its
condition is control.  

\subsubsection{-a, -c, and -delrange}

The active conditions are specified using {\em
-a} flags. The control conditions are specified using {\em -c} flags.
Slots for unspecified conditions are set to zero (all delay ranges).
The delay points to test are specified by the {\em -delrange} option
(if this option is omitted, then all delays are considered).  Columns
corresponding to delays outside of the range are set to zero
(for all conditions).

\subsubsection{-ircorr}

So far, we have only considered cases in which the slots of the
restriction table are either 0, +1, or -1.  One can view this
as a binary weighting in which a component is either completely
included or completely excluded.  The {\em -ircorr} flag allows
components to be weighted continuously.  This means that fractional
weights can be allowed in the restriction table.  These fractions are
determined by the ideal hemodynamic impulse response of equation
\ref{hideal.eqn}.  The weight at a particular column will be equal
to equation \ref{hideal.eqn} evaluated at the delay corresponding
to that column and using the $\Delta$ and $\tau$ specified by the
{\em -delta} and {\em -tau} flags.  This weighting is applied to
all specified conditions (ie, there is not a different $\Delta$ and
$\tau$ for different conditions).  Slots that are zero stay at zero,
and signs are preserved.

\subsection{Specifying a Test Type}

The restriction table determines which components of the hemodyanmic
responses will be included in the statistical test and how each will
be weighted. The table is completely specified by the the {\em -a, -c,
-delrange} and {\em -ircorr} options.  The {\em -testtype} and {\em
-pmin} options specify how these components will be combined.  In a t
test, all the non-zero components are summed (with appropriate weight
and sign).  In an F test, subsets of the non-zero slots may be summed
together (with appropriate weight and sign) followed by summing the
squares of the subtotals.  Subsets can be chosen across conditions
for the same delay (ie, collapse across conditions) or across delays
for the same condition (ie, collapse across delays) or both.  In some
cases the squaring of an element will make the distinction between
``active'' and ``control'' conditions irrelevant.

\subsubsection{ -testtype t}

All the elements specified by the non-zero slots are summed together
(with their appropriate sign).  The averages and variances are
combined resulting in a t statistic whose significance is then computed
with the t Test.

\subsubsection{ -testtype tm}

The conditions for each delay are summed together to create a separate
t value for each delay.  Significances are computed for each of these
values resulting in multiple p values for each voxel.  While the {\em
tm} test can report a significance for each delay at each voxel, the
addition of the {\em -pmin} flag will force only one significance per
voxel to be reported.

\subsubsection{ -testtype Fd}

Conditions for each delay are summed together (taking sign into
account).  These sums are squared and then added together to
create an F statistic.  This collapses across condition {\em before}
squaring and collapsing across delay, thus the ``active'' and 
``control'' labels are meaningful.  The difference between the {\em
Fd} option and the {\em Fm} option is that {\em Fd} collapses 
across delays after squaring resulting in a single F value (and
so single significance) per voxel.  The {\em Fm} option computes
an F value for each delay.

\subsubsection{ -testtype Fc}

Delays for each condition are summed together These sums are squared
and then added together to create an F statistic.  This collapses
across delays {\em before} squaring and collapsing across condition.
However, since all the delays within a condition will have the same
sign, the ``active'' and ``control'' labels are not meaningful.

\subsubsection{ -testtype Fcd}

Each non-zero component of the restriction table are squared {\em
before} adding them together to generate an F statistic. Note that
since the first step is a squaring operation, the sign is lost,
and the ``active'' and ``control'' labels become meaningless.

\subsection{Examples}

Most users think of hypotheses to be tested rather than restriction
tables to be derived.  This section describes how to go from one to
the other.  For example, consider the experiment in which there are
two non-null stimulus types (conditions 1 and 2) plus a fixation
(null) stimulus (condition 0) for a total of three conditions (ie,
$N_c=3$).  Assume that the the time window over which the hemodynamic
responses were averaged is 16 seconds for a TR of 2 seconds (ie, there
are 8 delay points per condition).  

We wish to evaluate the following hypotheses:
\begin{enumerate}
\item Condition 1 + Condition 2 are not significantly different than
Condition 0.
\item Condition 1 is not significantly different than Condition 1.
\end{enumerate}

Hypothesis 1 is a test of the ``main effect'' in which we want to
establish that the stimuli are indeed creating a measurable effect.
Here, we set the the active flags to {\em -a 1 -a 2} and a control
flag to {\em -c 0}.  In order to fill in the restriction table, we
need to specify a delay range.  Since we do not expect the hemodynamic
response to begin until about 2 seconds nor extend beyond 10 seconds,
we can restrict the test to these ranges: {\em -delrange 2 10}.  These
flags result in the following restriction table:\\

\begin{tabular}{|r||r|r|r|r|r|r|r|r|}\hline
& 0 TR & 1  TR & 2  TR & 3  TR & 4  TR & 5  TR & 6  TR & 7  TR\\ \hline \hline
Condition 0 & 0 &-1 &-1 &-1 &-1 &-1  & 0  & 0    \\ \hline
Condition 1 & 0 &+1 &+1 &+1 &+1 &+1  & 0  & 0    \\ \hline
Condition 2 & 0 &+1 &+1 &+1 &+1 &+1  & 0  & 0    \\ \hline
\end{tabular}\\

Hypothesis 2 is a tests one experimental conditions against the other.
For example, Condition 1 could be the visual presentation of a word
and Condition 2 could be the auditory presentation of a word.
Here, we set the the active flag to {\em -a 1} and the control
flag to {\em -c 2} and use the same delay range as above.  Note
that we could have chosen {\em -a 2} and {\em -c 1}.  This would not
make a difference in terms of the significance values but will make
a difference in terms of the resulting sign from a t test.  This
is meaningful in that it uncovers which voxels have relative increases in
blood flow and which have relative decreases.  The restriction table
for this test would be:\\

\begin{tabular}{|r||r|r|r|r|r|r|r|r|}\hline
& 0 TR & 1  TR & 2  TR & 3  TR & 4  TR & 5  TR & 6  TR & 7  TR\\ \hline \hline
Condition 0 & 0 &0 &0 &0 &0 &0  & 0  & 0    \\ \hline
Condition 1 & 0 &+1 &+1 &+1 &+1 &+1  & 0  & 0    \\ \hline
Condition 2 & 0 &-1 &-1 &-1 &-1 &-1  & 0  & 0    \\ \hline
\end{tabular}



\end{document}