\documentclass[10pt]{article}
\usepackage{amsmath}

%%%%%%%%%% set margins %%%%%%%%%%%%%
\addtolength{\textwidth}{1in}
\addtolength{\oddsidemargin}{-0.5in}
\addtolength{\textheight}{.75in}
\addtolength{\topmargin}{-.50in}

%%%%%%%%%%%%%%%%%%%%%%%%%%%%%%%%%%%%%%%%%%%%%%%%%%%%%%%%%%%%%%%%%
%%%%%%%%%%%%%%%%%%%%%%% begin document %%%%%%%%%%%%%%%%%%%%%%%%%%
%%%%%%%%%%%%%%%%%%%%%%%%%%%%%%%%%%%%%%%%%%%%%%%%%%%%%%%%%%%%%%%%%
\begin{document}

\begin{Large}
\noindent {\bf yakview} \\
\end{Large}

\noindent 
\begin{verbatim}
Comments or questions: analysis-bugs@nmr.mgh.harvard.edu
\end{verbatim}

\section{Introduction}
{\bf yakview} is a program for displaying the results of selective
averaging and statistical analysis of fMRI data on a slice-by-slice
basis. Specifically, it can be used to view images with multiple
planes, overlays, and time-courses of hemodynamic estimates for
multiple conditions in a point-and-click environment. Requires matlab
5.2 or higher.\\

Note: in previous versions (prior to 3/1/2000), yakview would, by
default, assume that significances were stored as the natural log of
the significance values.  It now defaults to log10 of the significance
values.

\section{Usage}
Typing yakview at the command-line without any options will give the
following message:\\ 

\begin{small}
\begin{verbatim}
USAGE: -i imgstem -sn slicenum -p sigstem -th threshold -f pformat -h hdrstem
    -i imgstem: stem of image to view
    -sn slice number
  Options:
    -p sigstem:       stem of stat map to overlay
    -thmin threshold: min threshold for stat overlay (2)
    -thmax threshold: max threshold for stat overlay (7)
    -h hdrstem: stem of hemodynamic averages
    -r rawstem: stem of raw data to plot
    -nskip n  : skip when displaying raw timecourses
    -off offsetstem: stem of offset volume (default is hdrstem-offset)
\end{verbatim}
\end{small}

\section{Command-line Arguments}

Note: there are two basic ways to invoke yakview.  The first is to
specify the entire name of a slice (including slice number and
extension). The second is to provide the stem and the slice number as
separate arguments.\\

\noindent
{\bf -i instem}: stem of the input volume for a single in {\em bfile format}. 
This can actually be either a stem or a full file name.  If it is a
stem, then it must be accompanied by the {\em -sn } flag to indicate
the slice number to use.\\

\noindent
{\bf -sn slicenumber}: the slice number of the stem(s) to view.  This
must be omitted if full file names are specified in the {\em -i}, {\em
-h}, and {\em -p} flags.  All the flags must specify either a full
file name or a stem; no mixing is allowed.\\

\noindent
{\bf -h hdrstem}: name of file in which the selective averages (ie,
hemodynamic responses) are stored.  This can actually be either a stem
or a full file name.  If it is a stem, then it must be accompanied by
the {\em -sn } flag to indicate the slice number to use. It is
expected that there will also be a data file called
``hdrstem.dat''. \\

\noindent
{\bf -r rawstem}: name of raw data file.  Hitting ``r'' inside the
yakview window will bring up a plot of the raw data at the active
voxel. \\

\noindent
{\bf -p sigstem}: name of the file in which the overlay data are
stored (in bfile format). This can actually be either a stem or a full
file name.  If it is a stem, then it must be accompanied by the {\em
-sn } flag to indicate the slice number to use.  The actual values in
the volume have no inherent meaning to yakview.  The color given to
overlay voxels is dependent upon the values of the minium and maximum
threshold (see -thmin and -thmax).  \\

\noindent
{\bf -f format}: obsolete.\\

\noindent
{\bf -thmin threshold}: specifies the minimum value that an overlay
voxel must achieve in order to be displayed as a color. This
is currently the only way to set the display threshold.\\

\noindent
{\bf -thmax threshold}: specifies the maximum value to map in the
color scale.  Voxels beyond this maximum will be displayed in the
color of maximum threshold.\\

\noindent
{\bf -nskip n}: specifies the number of initial data points to skip
when displaying raw time courses.\\

\noindent
{\bf -off offsetstem}: specifies the stem of an offset (or baseline)
file.  This volume indicates the factor (at each voxel) to use when
computing percent signal change (and so it is only relevant with the
-h option).  If {\em offsetstem} is unspecified, yakview will look for
a volume with stem {\em instem-offset} (which is automatically
produced with selxavg).\\

\section{General Usage}

In its most sophisticated invocation, yakview can be used to overlay a
significance map onto a co-planar structural image and to view the
hemodynamic responses at a voxel in a separate window by clicking on
the voxel in the image.  Manipulation of the data in both the image
window and the HDR window relies heavily on keypress commands.  A help
menu can be displayed for either window by clicking in the window and
then pressing the ``h'' key.

\subsection{The Image Window}

Upon invocation, only the image window is displayed. It will show the
underlying structural in gray scale and super-threshold statistics in
color.  Initially, only the voxels with statistics with a positive sign
are displayed.  Hitting the ``t'' key once will cause only those with
negative sign to be displayed. Hitting the ``t'' key again will
display the absolute value. Hitting the ``t'' key again will bring the
sign back to positive only.  Yakview keypress commands are:

\begin{table}[h]
\begin{center}
\begin{tabular}{|l|l|}\hline
+ & step to the next image plane \\ \hline
- & step to the previous image plane \\ \hline
b & bone scale  (instead of grey) \\ \hline
c & toggle crosshair \\ \hline
h & display help menu \\ \hline
i & invert underlay image\\ \hline
g & view hemodynamic averages \\ \hline
m & toggle marker \\ \hline
p & pink scale  (instead of grey) \\ \hline
q & exit yakview \\ \hline
r & view raw time courses \\ \hline
t & change overlay sign to view (+,-,+/-) \\ \hline
z & toggle zoom state \\ \hline
\end{tabular}
\end{center}
\end{table}

\subsection{The Hemodynamic Response Window}

To bring up the hemodyanmic response window (aka $HDRView$), click in
the image window and then press the ``g'' key.  Note that the
hemodynamic response {\em cannot} be viewed if the raw data was fit to
an ideal response. HDRView keypress commands are:

\begin{table}[h]
\begin{center}
\begin{tabular}{|l|l|}\hline
d & toggle subtraction of condition 0 \\ \hline
e & toggle display standard error bars \\ \hline
h & display help menu \\ \hline
n & toggle display of condition n (n=0,1,\ldots) \\ \hline
p & toggle dipslay of percent signal change \\ \hline
q & close HDRView \\ \hline
s & toggle display standard deviation bars \\ \hline
v & save graph as ascii file \\ \hline
z & subtract prestim avg for all conditions\\ \hline
\end{tabular}
\end{center}
\end{table}


\end{document}
