\documentclass[10pt]{article}
\usepackage{amsmath}

%%%%%%%%%% set margins %%%%%%%%%%%%%
\addtolength{\textwidth}{1in}
\addtolength{\oddsidemargin}{-0.5in}
\addtolength{\textheight}{.75in}
\addtolength{\topmargin}{-.50in}

%%%%%%%%%%%%%%%%%%%%%%%%%%%%%%%%%%%%%%%%%%%%%%%%%%%%%%%%%%%%%%%%%
%%%%%%%%%%%%%%%%%%%%%%% begin document %%%%%%%%%%%%%%%%%%%%%%%%%%
%%%%%%%%%%%%%%%%%%%%%%%%%%%%%%%%%%%%%%%%%%%%%%%%%%%%%%%%%%%%%%%%%
\begin{document}

\begin{Large}
\noindent {\bf mkmosaic16} \\
\end{Large}

\noindent 
\begin{verbatim}
Comments or questions: analysis-bugs@nmr.mgh.harvard.edu
\end{verbatim}

\section{Introduction}
{\bf mkmosiac16} converts 16 slices (stored in bfile format) into a $4
\times 4$ mosaic stored in a single bfile.  If each slice has multiple
planes, then each plane is also mosaiced and stored as a sepearate
plane in the mosaic file.  There is also an option to equalize the
distribution of voxel intensities so as to improve the contrast (good
for structurals). Requires matlab 5.2 or higher.\\

\section{Usage}
Typing mkmosaic16 at the command-line without any options will give the
following message:\\ 

\begin{small}
\begin{verbatim}
USAGE: <csh> mkmosaic16 instem <-o outstem> <-firstslice n> <-heq>
   instem - stem of the 16 input slices
   <-o outstem> - stem of mosaic (default is instem)
   <-firstslice int> - first slice number of the 16 slices 
              (default is 0)
   <-heq> - implement histogram equalization on each mosaic
            (default is no equalization)
\end{verbatim}
\end{small}

\section{Command-line Arguments}

\noindent
{\bf -i instem}: stem of the 16 input slices in {\em bfile format}. \\ 

\noindent
{\bf -o outstem}: optional stem of the output mosaic file.  If not
specified, the outstem is set to the intsem.\\

\noindent
{\bf -firstslice slicenumber}: the slice number of the first slice to
include in the mosaic. \\

\noindent
{\bf -h heq}: implement histogram equalization.  This can greatly
improve the image contrast for structural images.  Note that this
alters the intensity values of the image, and so it should only be
used for images in which the actual intensity values are of no
importance (eg, in structural images).\\


\section{Output}

The mosaic file will be called $outstem\_mos.bxxxxx$ where xxxxx is
either ``float'' or ``short'' depending upon the extension of the
input slices. If outstem is not specified, it is set to instem.


\end{document}
